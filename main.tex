\documentclass{article}
\usepackage{graphicx} % Required for inserting images
\usepackage[T2A] {fontenc} 

\title{Design document}
\author{Кирилл Бобровицкий, Максим Бобровицкий\\
Даниил Смирнов, Руслан Мухаметшин, Артем Антонов}
\date{November 2024}

\begin{document}

\maketitle

\tableofcontents
\newpage

\section{Введение}


\newpage

\section{Концепция}
\subsection{Введение}

\subsection{Жанр и аудитория}
Жанр: рогалик(rogue-like), приключенческий экшн.\\  
Аудитория: 12+

\subsection{Основные особенности игры}

\subsection{Описание игры}


\subsection{Предпосылки создания}
Личностный рост: Путь главного героя от простого жителя Междугорья до спасителя мира может стать источником глубокого личностного роста и трансформации персонажа. Это привлечет игроков, которые ценят развитие персонажей и сложные сюжеты.
Идентификация с героем: Уникальное положение главного героя как последнего незараженного человека делает его особенно привлекательным для игроков, позволяя им почувствовать себя частью чего-то великого и важного.
Эмоциональный резонанс: История главного героя, его борьба и жертвы ради спасения мира вызывают сильные эмоции у игроков, что способствует более глубокому погружению в игру.

\subsection{Платформа}
Windows, Mac, Linux

\newpage

\section{Функциональная спецификация}
\subsection{Принципы игры}
\subsubsection{Суть игрового процесса}

\subsubsection{Ход игры и сюжет}

\subsection{Физическая модель}

\subsection{Персонаж игрока}

\subsection{Элементы игры}

\subsection{«Искусственный интеллект»}

\subsection{Многопользовательский режим}


\subsection{Интерфейс пользователя}
\subsubsection{Блок-схема}


\subsubsection{Функциональное описание и управление}


\subsubsection{Объекты интерфейса пользователя}


\subsection{Графика и видео}
\subsubsection{Общее описание}


\subsubsection{Двумерная графика и анимация}


\subsubsection{Трехмерная графика и анимация}


\subsubsection{Анимационные вставки}


\subsection{Звуки и музыка}
\subsubsection{Общее описание}


\subsubsection{Звук и звуковые эффекты}


\subsubsection{Музыка}


\subsection{Описание уровней}
\subsubsection{Общее описание дизайна уровней}


\subsubsection{Диаграмма взаимного расположения уровней}


\subsubsection{График введения новых объектов}


\newpage

\section{Контакты}
Контактная информация для обратной связи.
\end{document}
